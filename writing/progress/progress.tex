\documentclass[12pt,english]{article}
\usepackage{fullpage,enumitem,amsmath,amssymb,graphicx}

% Some macros for your convenience
\newcommand\bbR{\ensuremath{\mathbb{R}}} % Real numbers
\newcommand\bbZ{\ensuremath{\mathbb{Z}}} % Integers
\newcommand\bbE{\ensuremath{\mathbb{E}}} % Expectation
\DeclareMathOperator*{\tr}{tr} % Trace
\DeclareMathOperator*{\diag}{diag} % Diagonal matrix
\DeclareMathOperator*{\sign}{sign} % Sign
\DeclareMathOperator*{\var}{Var} % Variance
\DeclareMathOperator*{\cov}{Cov} % Covariance
\newcommand{\1}{\mathbb{I}} % Indicator 

\title{
{\large CS229T/STATS231 Winter 2014 -- Project Progress Report }
}

\author{ \large
Awni Hannun \\
\texttt{awni@stanford.edu}
\and
Peng Qi \\
\texttt{pengqi@stanford.edu}
}
\date{}

\begin{document}
\maketitle

\subsubsection*{Introduction}
We attempt to prescribe a stochastic optimization procedure for the typical
multilayer neural network objective. We plan to do this by developing a deeper
understanding of the most typical objective(s) and architectures used in this
setting. We apply this analysis to motivate the selection of optimization
procedures we study and attempt to give a thorough comparison of a few
competing algorithms using existing theory and empirical results. 

\subsubsection*{Experimental Setup}

In most experiments performed, we use a standard multilayer neural network with
dense connections between each layer of hidden activations. The number of
hidden layers and size of the networks vary; for the MNIST experiments
described below we use 2 hidden layers with 200 units each. The rectified
linear nonlinearity ($f(x) = \max\{.1x,x\}$) is applied at each hidden unit. Most
computation is accelerated using GPUs and the Gnumpy for python library. 

We use the MNIST handwritten digits dataset which consists of 60,000 training
images and 10,000 test images. The images are all $28\times28$ pixels; however,
prior to training most networks, we project the data onto the 100 principal
components with standard PCA and use this as the input. This is primarily done
for computational efficiency, decreasing the total number of parameters in the
network. During optimization the network is usually trained for 30 epochs
(passes over the training data), randomly permuting the order of the training
examples to the network.

\subsubsection*{Understanding the Objective}

In order to gain insight about the objective function of neural networks, we
attempt to visualize them. To do this, we ran a set of experiments with models
comparably smaller than those used in our optimization experiments.  The
networks have an input dimension of 50 and hidden layers with 16 units each. We
try to limit the model size in order to efficiently train the networks for
repeated runs and to save the trace of model parameters and gradients as the
training progresses (we use these to visualize the cost function).

In these experiments we vary the number of hidden layers from zero to five and
the minibatch size in $\{120,600,6000,60000\}$.  While adding hidden layers is
known to help increase the expressivity of the model, we attempt to investigate
if more layers also makes the objective function of the network more poorly
conditioned. Further we wish to investigate how decreasing minibatch size degrades
our approximation to the objective function.

We run SGD for 1,000 iterations with a fixed learning rate, saving the
objective, parameters and gradients at each iteration. For each model setting,
we repeat the network training starting from 20 independent random initializations and
use these traces for our visualization.

To visualize the cost function, we compute the principal components of the
covariance matrix of the gradient traces across the 20 trials for each setting,
and project the model parameters into the subspace of the leading components.
The logic behind this is to characterize the landscape of the cost function in
the direction where the model parameters (despite initialization) travel the
most. 

Although the visualizations do not convey much information beyond models
without any hidden layers, from Figure \ref{fig:cost_layer} we do get a rough
sense that the nolinearity of the cost function grows as the number of hidden
layers increases. We see that the random initialization points (the red end of
each trace) are more concentrated under linear projection for the shallow
models than for the deep models. Another interesting phenomenon is that the
objective deviates further from the ``true'' objective as evaluated on the full
training set as the minibatch size decreases (Figure \ref{fig:cost_minibatch}).
This may be worth investigating further in order to characterize the quality of
the approximation to the objective given by a single minibatch. 

\section{Optimization}

We consider two families of stochastic optimization algorithms. Traditionally
used in optimizing deep neural networks, the first class is that of momentum
based methods which have been generalized and included in the broader known
group of accelerated gradient algorithms \cite{sutskever_2013}. The second
class includes variations on the adaptive gradient algorithms as presented in
\cite{duchi_2011}.

\subsection{Accelerated Gradient}

The accelerated gradient methods are a generalization of the basic stochastic
gradient descent (SGD) optimization procedure. Let $\ell(\theta)$ be the loss
function of the network with parameters $\theta$. We use accelerated SGD
methods that update the parameters at time $t$ according to
\begin{equation}
\label{nag}
\begin{aligned}
&v_{t+1} = \mu v_t - \eta \nabla \ell (\theta_t + \gamma v_t) \\
&\theta_{t+1} = \theta_t + v_{t+1}
\end{aligned}
\end{equation}
where $\eta$ is the learning rate, and the parameter $\mu$, known as the
momentum parameter, dictates how much gradient history we take into account at
every update. If we set $\mu = 0$ we recover plain SGD, and setting $\mu = 1$
uses the full gradient history at every update.  The $\gamma$ parameter is
either active and set as $\gamma = \mu$ or inactive and set as $\gamma = 0$.
When $\gamma$ is inactive, we have typical SGD with ``momentum'', also known as
classical momentum (CM). On the other hand, when $\gamma$ is active the
procedure is known as Nesterov's accelerated graient (NAG).

A common belief is that network objectives likely suffer from long narrow
ravines leading towards local optima surrounded with walls of high curvature.
This hypothesis motivates the use of momentum to encourage persistent
directions of travel along the basin and suppress unwanted oscillation.

\subsection{Adaptive Gradient}
\label{adagrad}

We study modifications to the adaptive gradient algorithm (AdaGrad) with the
diagonal preconditioning matrix $G$. The update can be written as
\begin{equation}
\begin{aligned}
&G_{t+1} = G_{t} + \diag(\nabla \ell (\theta_t))^2 \\
&\theta_{t+1} = \theta_t - G_{t+1}^{-1/2} \nabla \ell(\theta_t)
\end{aligned} 
\end{equation}

In some sense AdaGrad achieves a similar affect as NAG by penalizing
oscillating directions in which we take large steps and encouraging directions
with small but consistent gradients. However, in other aspects the optimization
routines behave quite differently. A simple property of AdaGrad which as we
show later can drasitcally affect optimization is the nondecreasing
monotinicity of the matrix $G$. This leads AdaGrad to penalize large yet
consistent directions of travel in the optimization process. This proves
locally favorable in most of the models studied in the next section, but
globally the behaviour results in better performance of NAG routine over the
vanilla AdaGrad algorithm.

This motivates two simple variations of the AdaGrad algorithm. The first is to
replace the square root on the matrix $G_t$ with a function which grows less
quickly (e.g. cube root). The second is to decay $G_t$ by a factor of $\gamma
\in (0,1]$ before including it in the update for $G_{t+1}$. This is close to
the AdaDelta algorithm presented in \cite{zeiler_2012}, but there they take a
convex combination of the two terms in the update of $G_{t+1}$ whereas here we
only decay the gradient history term.

% Discuss convergence properties of each method?


\subsubsection*{Plans}

Further plans for our project include:

\begin{enumerate}

\item Gain better understanding of the network objective(s) by using second
order information.  For example for small $n$, we can estimate rank $n$
approximations to the Hessian of the objective to better understand local
curvature.

\item Evaluate the same empirical studies on a slightly harder to fit dataset.
We plan to move to CIFAR-100 next.

\item Compare the NAG, AdaGrad and other variations of the two optimization
procedures to convergence rates given in the literature.

\item Give an evaluation and discussion on generalization performance.

\end{enumerate}

\subsubsection*{Appendix}

\showthe\columnwidth

\bibliography{refs}{}
\bibliographystyle{plain}

\end{document}

